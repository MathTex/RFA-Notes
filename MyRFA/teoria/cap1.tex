%!TEX root = ../main.tex

\chapter{Set Theory}
\thispagestyle{empty}

\section*{Equipotent, finite/infinite, countable/uncountable sets, cardinality of continoum}

Let $X,Y$ be sets.

\begin{defn}[Equipotent sets]$\\$
$X,Y$ are equipotent if there exists a bijection $f:X\to Y$ (1-1 injective + onto surjective).
\end{defn}

If $X,Y$ are equipotent, then they have the same cardinality. On the other hand, $X$ has cardinality $\geq$ than $Y$ if there exists $f:X\to Y$ onto. For example, for
\begin{equation*}
X= 
\begin{pmatrix}
 1\\
 2\\
 3
\end{pmatrix} \qquad Y=\begin{pmatrix}
 1\\
 2
\end{pmatrix}
\end{equation*}

exists $f:X\to Y$ s.t. $\forall y\in Y\ \exists x\in X$ s.t. $f(x)=y$ ($f$ \emph{takes} all the elements of the codomain), but doesn't exist $g:Y\to X$ s.t. $\forall x\in X\ \exists y\in Y$ s.t. $g(y)=x$ ($g$ doesn't \emph{take} all the elements of the codomain).

\begin{defn}[Finite/infinite sets]$\\$
$X$ is finite if it is equipotent to $Y=\{1,2,...,k\}$ for some $k\in\NN$. $X$ is infinite otherwise.
\end{defn}

\begin{prp}
$X$ is infinite iff it is equipotent to a proper subset, i.e. if exists a bijection between $X$ and one of his subsets.
\end{prp}

For example, between the integers set $\ZZ=\{0,\pm1,\pm2,...\}$ and the even integers set $\{0,\pm2,\pm4,...\}$ there exists $f$ s.t. $f(z)=2z$ which is a bijection.

\begin{defn}[Countable/uncountable (infinite) sets]$\\$
$X$ inifinite is countable if it is equipotent to $\NN$. It is uncountable otherwise, in which case is more than countable (countable sets are the "smallest" among infinite sets).
\end{defn}

\begin{defn}[Cardinality of continoum]$\\$
$X$ has the cardinality of continoum if it is equipotent to $\RR$. Any such set is uncountable.
\end{defn}

For example:
\begin{itemize}
	\item $\NN,\ZZ,\QQ$ are countable
	\item $\RR,\RR^N,(0,1),(0,1)^N$ have the cardinality of continoum
	\item countable unions of countable sets are countable
\end{itemize}













