%!TEX root = ../main.tex

\chapter{Set Theory}
\thispagestyle{empty}

\section{Equipotent, finite/infinite, countable/uncountable sets, cardinality of continoum} % (fold)
\label{sec:equipotent_finite_infinite_countable_uncountable_sets_cardinality_of_continoum}

Let $X,Y$ be sets.

\begin{defn}[Equipotent sets]$\\$
$X,Y$ are equipotent if there exists a bijection $f:X\to Y$ (1-1 injective + onto surjective).
\end{defn}

If $X,Y$ are equipotent, then they have the same cardinality. On the other hand, $X$ has cardinality $\geq$ than $Y$ if there exists $f:X\to Y$ onto. For example, for
\begin{equation*}
X= 
\begin{pmatrix}
 1\\
 2\\
 3
\end{pmatrix} \qquad Y=\begin{pmatrix}
 1\\
 2
\end{pmatrix}
\end{equation*}

exists $f:X\to Y$ s.t. $\forall y\in Y\ \exists x\in X$ s.t. $f(x)=y$ ($f$ \emph{takes} all the elements of the codomain), but doesn't exist $g:Y\to X$ s.t. $\forall x\in X\ \exists y\in Y$ s.t. $g(y)=x$ ($g$ doesn't \emph{take} all the elements of the codomain).

\begin{defn}[Finite/infinite sets]$\\$
$X$ is finite if it is equipotent to $Y=\{1,2,...,k\}$ for some $k\in\NN$. $X$ is infinite otherwise.
\end{defn}

\begin{prp}
$X$ is infinite iff it is equipotent to a proper subset, i.e. if exists a bijection between $X$ and one of his subsets.
\end{prp}

For example, between the integers set $\ZZ=\{0,\pm1,\pm2,...\}$ and the even integers set $\{0,\pm2,\pm4,...\}$ there exists $f$ s.t. $f(z)=2z$ which is a bijection.

\begin{defn}[Countable/uncountable (infinite) sets]$\\$
$X$ inifinite is countable if it is equipotent to $\NN$. It is uncountable otherwise, in which case is more than countable (countable sets are the "smallest" among infinite sets).
\end{defn}

\begin{defn}[Cardinality of continoum]$\\$
$X$ has the cardinality of continoum if it is equipotent to $\RR$. Any such set is uncountable.
\end{defn}

For example:
\begin{itemize}
	\item $\NN,\ZZ,\QQ$ are countable
	\item $\RR,\RR^N,(0,1),(0,1)^N$ have the cardinality of continoum
	\item countable unions of countable sets are countable
\end{itemize}

% section equipotent_finite_infinite_countable_uncountable_sets_cardinality_of_continoum (end)

\section{Families of subsets} % (fold)
\label{sec:families_of_subsets}

Let $X$ be a set.

\begin{defn}[Power set]$\\$
The power set of $X$, i.e. the set of all subsets of $X$, is
\begin{equation*}
    \Pc(X)=\gr{Y\,:\,Y\subset X}
\end{equation*}
It is sometimes denoted as $2^X$.
\end{defn}

The power set has cardinality strictly bigger than $X$. For example, $\Pc(\NN)$ has the cardinality of continoum.

\begin{defn}[Family of subsets]$\\$
A family, or collection, of subsets of $X$ is just $\Cc\subset\Pc(X)$. Tipically, a family of subsets (induced by $I\subset\RR$ set of indexes) is $\Cc=\gr{E_i}_{i\in I}$ where $E_i\subset X$ $\forall i\in I$.
\end{defn}

For example, $\left\{E_1,E_2,E_3\right\}$ is a family of subsets.

\begin{defn}[Union and intersection]$\\$
Given a family of sets $\left\lbrace E_i \right\rbrace_{i \in I} \subset \mathcal{P}(X)$, will often be considered
\begin{align*}
\bigcup_{i \in I} E_i &= \gr{x \in X\,:\, \exists i \in I \text{ s.t. } x \in E_i} \\ 
\bigcap_{i \in I} E_i &= \gr{x \in X\,:\,x \in E_i\ \forall i \in I}  
\end{align*}

$\gr{E_i}$ is said to be (pairwise) disjoint if $E_i \cap E_j = \emptyset$ $\forall i \not = j$.
\end{defn}

\begin{exa}[Standard topology of \texorpdfstring{$\RR$}{C}]$\\$
Given $X=\RR$ $\td{\text{or }\RR^N}$, the standard/euclidian topology of $\RR$ $\td{\text{or }\RR^N}$ is $\Tc=\gr{E\subset X\,:\,E\text{ is open}}$, i.e. it is the family of all open subsets of $X$. \\
More generally, this can be defined in metric spaces $\td{X,d}$ where $X$ is a set and $d$ a distance between $x,y\in X$.

Some properties of $\Tc$:
\begin{itemize}
    \item $\varnothing,X\in\Tc$
    \item finite intersection of open sets is open [$\circledast$]
    \item any (finite/infinite, countable/uncountable, ...) union of open sets is open [$\circledcirc$]
\end{itemize}
\end{exa}

\begin{defn}[Covering and subcovering]$\\$
$\gr{E_i}_{i\in I}$ is a covering of $X$ if $X=\bigcup_{i\in I} E_i$. Any subfamily $\gr{E_i}_{i\in J, J\subset I}$ is a subcovering if it is a covering.
\end{defn}

% section families_of_subsets (end)

\newpage

\section{Sequences of sets} % (fold)
\label{sec:sequences_of_sets}

A sequence is just a family of subsets where $I\equiv \NN$, e.g. $\gr{E_n}_{n\in \NN}$. 

\begin{defn}[Monotone sequences]$\\$
$\gr{E_n}$ is increasing (not decreasing), $\gr{E_n}\nearrow$, if $E_n\subset E_{n+1}\ \forall n\in\NN$. On the other hand, $\gr{E_n}$ is decreasing (not increasing), $\gr{E_n}\searrow$, if $E_{n+1}\subset E_{n}\ \forall n\in\NN$. If $\gr{E_n}$ is increasing/decreasing then it is monotone.
\end{defn}

For example, given $X=\RR$ and $E_n\displaystyle=\td{-\frac{1}{n},1+\frac{1}{n}}$ for $n\geq 1$, we can say that $E_n$ is a monotone decreasing sequence:
\fg{0.3}{endrc}

But what is $\bigcap_{n=1}^\infty E_n$? We know that 
\begin{equation*}
    \bigcap_{n=1}^\infty E_n=[0,1]
\end{equation*}

and this is an infinite intersection of open sets (this does not disagree with the prop $\circledast$). This type of intersection is called "G$\delta$-set": a countable intersection of open sets.

Similarly, $E_n=\displaystyle\qu{a+\frac{1}{n},b-\frac{1}{n}}$, a<b, is increasing and 
\begin{equation*}
    \bigcup_{n=1}^\infty E_n=(a,b)
\end{equation*}
is called "F$\sigma$-set": a countable union of closed sets (doesn't disagree with $\circledcirc$).

\begin{defn}[lim sup and lim inf]$\\$
Let $\gr{E_n}_{n\in\NN}\subset\Pc$. We define
\begin{equation*}
    \limsup_{n} E_n := \bigcap_{n = 1}^{\infty} \left(\bigcup_{k = n}^{\infty} E_k\right)
    \qquad
    \liminf_{n} E_n := \bigcup_{n = 1}^{\infty} \left(\bigcap_{k = n}^{\infty} E_k\right)
\end{equation*}

If these two sets are equal
\begin{equation*}
    \limsup_n E_n = \liminf_n E_n = \lim_n E_n = F
\end{equation*}
then $F$ is the limit of the succession.
\end{defn}

Take note that $\gr{E_n}\nearrow \td{\searrow} \Rightarrow \exists \lim_n E_n=\bigcup_n E_n \td{\bigcap_n E_n}$.

% section sequences_of_sets (end)



































