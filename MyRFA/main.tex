\documentclass[10pt,a4paper,twoside]{book}

%%%%%%%%%%%%%%%%%%%%%%%%%%%%%%%%%%%%%%%%%
% Template Dispense
% Autore: Teo Bucci
%%%%%%%%%%%%%%%%%%%%%%%%%%%%%%%%%%%%%%%%%

%---------------------------
% FONTS AND LANGUAGE
%---------------------------

\usepackage[T1]{fontenc}
\usepackage[utf8]{inputenc}
\usepackage[italian]{babel}

%---------------------------
% PACKAGES
%---------------------------

\usepackage{dsfont} % for using \mathds{1} characteristic function
\usepackage{amsmath, amssymb, amsthm} % amssymb also loads amsfonts
\usepackage{latexsym}

\usepackage{booktabs}
\usepackage{pgfplots}
\usepackage{tikz}
\usetikzlibrary{
  positioning,
  shapes.misc,
  intersections,
  shapes.symbols,
  patterns,
  fadings,
  shadows.blur,
  decorations.pathreplacing
}
\usepackage{mathdots}
\usepackage{cancel}
\usepackage{color}
\usepackage{siunitx}
\usepackage{array}
\usepackage{multirow}
\usepackage{makecell}
\usepackage{tabularx}
\usepackage{caption}
\captionsetup{belowskip=12pt,aboveskip=4pt}
\usepackage{subcaption}
\usepackage{placeins} % \FloatBarrier
\usepackage{flafter}  % The flafter package ensures that floats don't appear until after they appear in the code.
\usepackage[shortlabels]{enumitem}
\usepackage[italian]{varioref}
\renewcommand{\ref}{\vref}

\newcommand{\quadretti}[3]{ 
\begin{tikzpicture}
\tikzset{normal lines/.style={gray, very thin}} 
\draw[style=normal lines,step=#1] (0,0) grid +(#2,#3); 
\end{tikzpicture}}

% \quadretti{4mm}{160mm}{88mm} % misura lato quadratino, larghezza, altezza del box di quadratini

%---------------------------
% INCLUSIONE FIGURE
%---------------------------

\usepackage{import}
\usepackage{pdfpages}
\usepackage{transparent}
\usepackage{xcolor}
\usepackage{graphicx}
\graphicspath{ {./images/} } % Path relative to the main .tex file
\usepackage{float}

\newcommand{\fg}[3][\relax]{%
  \begin{figure}[H]%[htp]%
    \centering
    \captionsetup{width=0.7\textwidth}
      \includegraphics[width = #2\textwidth]{#3}%
      \ifx\relax#1\else\caption{#1}\fi
      \label{#3}
  \end{figure}%
  \FloatBarrier%
}
%\usepackage[labelformat=empty]{caption}

%\captionsetup{belowskip=0pt,aboveskip=0pt}

\let\origtopsep\topsep
\newenvironment{hfigure}[1][\origtopsep]{\begingroup\captionsetup{belowskip=0pt,aboveskip=2pt}
\setlength{\topsep}{#1}\begin{center}}
{\end{center}\endgroup}

%---------------------------
% PARAGRAPHS AND LINES
%---------------------------

\usepackage[none]{hyphenat} % no hyphenation

\emergencystretch 3em % to prevent the text from going beyond margins

\usepackage[skip=0.2\baselineskip+2pt]{parskip}

% \renewcommand{\baselinestretch}{1.5} % line spacing

%---------------------------
% HEADERS AND FOOTERS
%---------------------------

\usepackage{fancyhdr}

\fancypagestyle{toc}{%
\fancyhf{}%
\fancyfoot[C]{\thepage}%
\renewcommand{\headrulewidth}{0pt}%
\renewcommand{\footrulewidth}{0pt}
}

\fancypagestyle{fancy}{%
\fancyhf{}%
\fancyhead[RE]{\nouppercase{\leftmark}}%
\fancyhead[LO]{\nouppercase{\rightmark}}%
\fancyhead[LE,RO]{\thepage}%
\renewcommand{\footrulewidth}{0pt}%
\renewcommand{\headrulewidth}{0.4pt}
}

% Removes the header from odd empty pages at the end of chapters
\makeatletter
\renewcommand{\cleardoublepage}{
\clearpage\ifodd\c@page\else
\hbox{}
\vspace*{\fill}
\thispagestyle{empty}
\newpage
\fi}

\usepackage{nonumonpart}

%---------------------------
% CUSTOM
%---------------------------

\usepackage{xspace}
\newcommand{\latex}{\LaTeX\xspace}
\newcommand{\tex}{\TeX\xspace}

\newcommand{\Tau}{\mathcal{T}}
\newcommand{\Ind}{\mathds{1}} % indicatrice

\newcommand{\transpose}{^{\mathrm{T}}}
\newcommand{\complementary}{^{\mathrm{C}}} % alternative ^{\mathrm{C}} ^{\mathrm{c}} ^{\mathsf{c}}
\newcommand{\degree}{^\circ\text{C}} % simbolo gradi

\newcommand{\notimplies}{\mathrel{{\ooalign{\hidewidth$\not\phantom{=}$\hidewidth\cr$\implies$}}}}
\newcommand{\questeq}{\overset{?}{=}} % è vero che?

\newcommand{\indep}{\perp \!\!\! \perp} % indipendenza
\newcommand{\iid}{\stackrel{\mathrm{iid}}{\sim}}
\newcommand{\event}[1]{\emph{``#1''}} % evento

% variazioni del simbolo "="
\newcommand{\iideq}{\overset{\text{\tiny iid}}{=}}
\newcommand{\ideq}{\overset{\text{\tiny id}}{=}}
\newcommand{\indepeq}{\overset{\perp \!\!\! \perp}{=}}

\newcommand{\boxedText}[1]{\noindent\fbox{\parbox{\textwidth}{#1}}}

\renewcommand{\emptyset}{\varnothing}
\renewcommand{\tilde}{\widetilde}
\renewcommand{\hat}{\widehat}

\DeclareMathOperator{\sgn}{sgn}
\DeclareMathOperator{\Var}{Var}
\DeclareMathOperator{\Cov}{Cov}
\DeclareMathOperator*{\rank}{rank}
\DeclareMathOperator*{\eig}{eig}
\DeclareMathOperator{\tr}{tr}
%\DeclareMathOperator{\Grad}{grad}
%\DeclareMathOperator{\Div}{div}
\DeclareMathOperator{\Span}{span}
\let\Re\undefined  % redefine \Re
\DeclareMathOperator{\Re}{Re}
\let\Im\undefined  % redefine \Im
\DeclareMathOperator{\Im}{Im}
\DeclareMathOperator{\Ker}{Ker}
\DeclareMathOperator*{\argmin}{arg\,min}
\DeclareMathOperator*{\argmax}{arg\,max}
\DeclareMathOperator*{\esssup}{ess\ sup}
\DeclareMathOperator*{\essinf}{ess\ inf}
\DeclareMathOperator*{\supp}{supp}

\newcommand{\eps}{\varepsilon}
\renewcommand{\theta}{\vartheta}

% Per scrivere il numero e la data della lezione


\usepackage{mathtools} % Serve per i comandi dopo
%\DeclarePairedDelimiter{\abs}{\lvert}{\rvert} % absolute value
%\DeclarePairedDelimiter{\sca}{\langle}{\rangle} % scalar product
%\DeclarePairedDelimiter{\norm}{\lVert}{\rVert} % norm
\newcommand{\abs}[1]{\left\lvert #1 \right\rvert}
\newcommand{\norm}[1]{\left\lVert #1 \right\rVert}
\newcommand{\sca}[1]{\left\langle #1 \right\rangle}
\newcommand{\parteInf}[1]{\left\lfloor #1 \right\rfloor}
\newcommand{\parteSup}[1]{\left\lceil #1 \right\rceil}

\newcommand{\gr}[1]{\left\{ #1 \right\}}
\newcommand{\qu}[1]{\left[ #1 \right]}
\newcommand{\td}[1]{\left(#1 \right)}

% Bold
\renewcommand{\AA}{\mathbb A}
\newcommand{\BB}{\mathbb{B}}
\newcommand{\CC}{\mathbb{C}}
\newcommand{\DD}{\mathbb{D}}
\newcommand{\EE}{\mathbb{E}}
\newcommand{\FF}{\mathbb{F}}
\newcommand{\GG}{\mathbb{G}}
\newcommand{\HH}{\mathbb{H}}
\newcommand{\II}{\mathbb{I}}
\newcommand{\JJ}{\mathbb{J}}
\newcommand{\KK}{\mathbb{K}}
\newcommand{\LL}{\mathbb{L}}
\newcommand{\MM}{\mathbb{M}}
\newcommand{\NN}{\mathbb{N}}
\newcommand{\OO}{\mathbb{O}}
\newcommand{\PP}{\mathbb{P}}
\newcommand{\QQ}{\mathbb{Q}}
\newcommand{\RR}{\mathbb{R}}
\renewcommand{\SS}{\mathbb S}
\newcommand{\TT}{\mathbb{T}}
\newcommand{\UU}{\mathbb{U}}
\newcommand{\VV}{\mathbb{V}}
\newcommand{\WW}{\mathbb{W}}
\newcommand{\XX}{\mathbb{X}}
\newcommand{\YY}{\mathbb{Y}}
\newcommand{\ZZ}{\mathbb{Z}}

% Calligraphic
\newcommand{\Ac}{\mathcal{A}}
\newcommand{\Bc}{\mathcal{B}}
\newcommand{\Cc}{\mathcal{C}}
\newcommand{\Dc}{\mathcal{D}}
\newcommand{\Ec}{\mathcal{E}}
\newcommand{\Fc}{\mathcal{F}}
\newcommand{\Gc}{\mathcal{G}}
\newcommand{\Hc}{\mathcal{H}}
\newcommand{\Ic}{\mathcal{I}}
\newcommand{\Jc}{\mathcal{J}}
\newcommand{\Kc}{\mathcal{K}}
\newcommand{\Lc}{\mathcal{L}}
\newcommand{\Mc}{\mathcal{M}}
\newcommand{\Nc}{\mathcal{N}}
\newcommand{\Oc}{\mathcal{O}}
\newcommand{\Pc}{\mathcal{P}}
\newcommand{\Qc}{\mathcal{Q}}
\newcommand{\Rc}{\mathcal{R}}
\newcommand{\Sc}{\mathcal{S}}
\newcommand{\Tc}{\mathcal{T}}
\newcommand{\Uc}{\mathcal{U}}
\newcommand{\Vc}{\mathcal{V}}
\newcommand{\Wc}{\mathcal{W}}
\newcommand{\Xc}{\mathcal{X}}
\newcommand{\Yc}{\mathcal{Y}}
\newcommand{\Zc}{\mathcal{Z}}

% Bold Big Vector
\newcommand{\Av}{\mathbf{A}}
\newcommand{\Bv}{\mathbf{B}}
\newcommand{\Cv}{\mathbf{C}}
\newcommand{\Dv}{\mathbf{D}}
\newcommand{\Ev}{\mathbf{E}}
\newcommand{\Fv}{\mathbf{F}}
\newcommand{\Gv}{\mathbf{G}}
\newcommand{\Hv}{\mathbf{H}}
\newcommand{\Iv}{\mathbf{I}}
\newcommand{\Jv}{\mathbf{J}}
\newcommand{\Kv}{\mathbf{K}}
\newcommand{\Lv}{\mathbf{L}}
\newcommand{\Mv}{\mathbf{M}}
\newcommand{\Nv}{\mathbf{N}}
\newcommand{\Ov}{\mathbf{O}}
\newcommand{\Pv}{\mathbf{P}}
\newcommand{\Qv}{\mathbf{Q}}
\newcommand{\Rv}{\mathbf{R}}
\newcommand{\Sv}{\mathbf{S}}
\newcommand{\Tv}{\mathbf{T}}
\newcommand{\Uv}{\mathbf{U}}
\newcommand{\Vv}{\mathbf{V}}
\newcommand{\Wv}{\mathbf{W}}
\newcommand{\Xv}{\mathbf{X}}
\newcommand{\Yv}{\mathbf{Y}}
\newcommand{\Zv}{\mathbf{Z}}

% Bold Little Vector
\newcommand{\av}{\mathbf{a}}
\newcommand{\bv}{\mathbf{b}}
\newcommand{\cv}{\mathbf{c}}
\newcommand{\dv}{\mathbf{d}}
\newcommand{\ev}{\mathbf{e}}
\newcommand{\fv}{\mathbf{f}}
\newcommand{\gv}{\mathbf{g}}
\newcommand{\hv}{\mathbf{h}}
\newcommand{\iv}{\mathbf{i}}
\newcommand{\jv}{\mathbf{j}}
\newcommand{\kv}{\mathbf{k}}
\newcommand{\lv}{\mathbf{l}}
\newcommand{\mv}{\mathbf{m}}
\newcommand{\nv}{\mathbf{n}}
\newcommand{\ov}{\mathbf{o}}
\newcommand{\pv}{\mathbf{p}}
\newcommand{\qv}{\mathbf{q}}
\newcommand{\rv}{\mathbf{r}}
\newcommand{\sv}{\mathbf{s}}
\newcommand{\tv}{\mathbf{t}}
\newcommand{\uv}{\mathbf{u}}
\newcommand{\vv}{\mathbf{v}}
\newcommand{\wv}{\mathbf{w}}
\newcommand{\xv}{\mathbf{x}}
\newcommand{\yv}{\mathbf{y}}
\newcommand{\zv}{\mathbf{z}}

% differenziale
\newcommand{\dspace}{\ } % \, aggiunge un piccolo spazio
\newcommand{\de}{\mathrm{d}}
\newcommand{\dx}{\dspace \de x}
\newcommand{\dy}{\dspace \de y}
\newcommand{\dt}{\dspace \de t}
\newcommand{\dS}{\dspace \de S}
\newcommand{\ds}{\dspace \de s}
\newcommand{\dz}{\dspace \de z}
\newcommand{\dw}{\dspace \de w}
\newcommand{\du}{\dspace \de u}
\newcommand{\dvv}{\dspace \de v}
\newcommand{\db}{\dspace \de b}
\newcommand{\dteta}{\dspace \de \vartheta}
\newcommand{\dxi}{\dspace \de \xi}
\newcommand{\dxy}{\dspace \de x \de y}
\newcommand{\duv}{\dspace \de u \de v}
\newcommand{\dst}{\dspace \de s \de t}
\newcommand{\dP}{\dspace \de P}
\newcommand{\dPP}{\dspace \de \PP}
\newcommand{\dsig}{\dspace \de \sigma}
\newcommand{\dth}{\dspace \de \theta}
\newcommand{\deta}{\dspace \de \eta}
\newcommand{\dph}{\dspace \de \varphi}
\newcommand{\dxv}{\dspace \de \mathbf{x}}
\newcommand{\dSx}{\dspace \de \text{S}(x)}

\newcommand{\Grad}{\nabla}
\newcommand{\Div}{\mathrm{div}}
\newcommand{\Lap}{\Delta}
% \newcommand{\Dalem}{\Box} per l'eq delle onde?

\newcommand{\SDP}{(\Omega,\Ac,\PP)} % spazio di probabilità
\newcommand{\Omegaa}{\overline{\Omega}} % chiusura
\newcommand{\Cz}{\Cc^0}
\newcommand{\Cu}{\Cc^1}
\newcommand{\Cd}{\Cc^2}
\newcommand{\Lu}{\mathcal{L}^1}
\newcommand{\ld}{\ell^2}
\newcommand{\frp}{\partial_pQ_T}

\newcommand{\hod}[1]{^{\scriptscriptstyle\mathrm{#1}}} % per le derivate con exp romano

%\newcommand{\Log}{\text{Log}}

% spaziature https://tex.stackexchange.com/questions/438612/space-between-exists-and-forall
% questo aggiunge un piccolo spazio dopo \forall
\let\oldforall\forall
\renewcommand{\forall}{\oldforall \, }
% questo aggiunge un piccolo spazio dopo \exists
\let\oldexist\exists
\renewcommand{\exists}{\oldexist \: }
% questo aggiunge un comando \existsu per l'esiste ed è unico
\newcommand\existu{\oldexist! \: }

%---------------------------
% APPENDICE
%---------------------------

\usepackage[title,titletoc]{appendix}

%---------------------------
% THEOREMS
%---------------------------

\definecolor{grey245}{RGB}{245,245,245}

\newtheoremstyle{blacknumbox} % Theorem style name
{0pt}% Space above
{0pt}% Space below
{\normalfont}% Body font
{}% Indent amount
{\bf\scshape}% Theorem head font --- {\small\bf}
{.\;}% Punctuation after theorem head
{0.25em}% Space after theorem head
{\small\thmname{#1}\nobreakspace\thmnumber{\@ifnotempty{#1}{}\@upn{#2}}% Theorem text (e.g. Theorem 2.1)
%{\small\thmname{#1}% Theorem text (e.g. Theorem)
\thmnote{\nobreakspace\the\thm@notefont\normalfont\bfseries---\nobreakspace#3}}% Optional theorem note

\newtheoremstyle{unnumbered} % Theorem style name
{0pt}% Space above
{0pt}% Space below
{\normalfont}% Body font
{}% Indent amount
{\bf\scshape}% Theorem head font --- {\small\bf}
{.\;}% Punctuation after theorem head
{0.25em}% Space after theorem head
{\small\thmname{#1}\thmnumber{\@ifnotempty{#1}{}\@upn{#2}}% Theorem text (e.g. Theorem 2.1)
%{\small\thmname{#1}% Theorem text (e.g. Theorem)
\thmnote{\nobreakspace\the\thm@notefont\normalfont\bfseries---\nobreakspace#3}}% Optional theorem note

\newtheoremstyle{demo} % Theorem style name
{0pt}% Space above
{0pt}% Space below
{\normalfont}% Body font
{}% Indent amount
{\bf\scshape}% Theorem head font --- {\small\bf}
{.\;}% Punctuation after theorem head
{0.25em}% Space after theorem head
{\small\thmname{#1}\thmnumber{\@ifnotempty{#1}{}\@upn{#2}}% Theorem text (e.g. Theorem 2.1)
%{\small\thmname{#1}% Theorem text (e.g. Theorem)
\thmnote{\nobreakspace\the\thm@notefont\normalfont\bf\scshape\footnotesize{(#3)}}}% Optional theorem note

\newcounter{dummy}
\numberwithin{dummy}{chapter}

\newcounter{dummyNOT}
\numberwithin{dummyNOT}{chapter}

\theoremstyle{blacknumbox}
\newtheorem{theoremT}[dummy]{Th}
\newtheorem{corollaryT}[dummy]{Cor}
\newtheorem{lemmaT}[dummy]{Lemma}
\newtheorem{exerciseT}[dummyNOT]{Exer}

% Per gli unnumbered tolgo il \nobreakspace subito dopo {\small\thmname{#1} perché altrimenti c'è uno spazio tra Teorema e il ".", lo spazio lo voglio solo se sono numerati per distanziare Teorema e "(2.1)"
\theoremstyle{unnumbered}
\newtheorem*{remarkT}{Remark}
\newtheorem*{exampleT}{Ex}
\newtheorem*{propertyT}{Prop}
\newtheorem*{homeworkT}{Homework}
\newtheorem*{hintT}{Subtleties}
\newtheorem*{definitionT}{Def}

\theoremstyle{demo}
\newtheorem*{proofT}{Proof}

\RequirePackage[framemethod=default]{mdframed} % Required for creating the theorem, definition, exercise and corollary boxes

% orange box
\newmdenv[skipabove=7pt,
skipbelow=4pt,
rightline=true,
leftline=true,
topline=true,
bottomline=true,
linecolor=orange,
backgroundcolor=orange!0,
innerleftmargin=5pt,
innerrightmargin=5pt,
innertopmargin=5pt,
leftmargin=0cm,
rightmargin=0cm,
linewidth=1pt,
innerbottommargin=5pt]{oBox}

% green box
\newmdenv[skipabove=7pt,
skipbelow=4pt,
rightline=true,
leftline=true,
topline=true,
bottomline=true,
linecolor=green,
backgroundcolor=green!0,
innerleftmargin=5pt,
innerrightmargin=5pt,
innertopmargin=5pt,
leftmargin=0cm,
rightmargin=0cm,
linewidth=1pt,
innerbottommargin=5pt]{gBox}

% blue box
\newmdenv[skipabove=7pt,
skipbelow=4pt,
rightline=true,
leftline=true,
topline=true,
bottomline=true,
linecolor=blue,
backgroundcolor=blue!0,
innerleftmargin=5pt,
innerrightmargin=5pt,
innertopmargin=5pt,
leftmargin=0cm,
rightmargin=0cm,
linewidth=1pt,
innerbottommargin=5pt]{bBox}

% purple box
\newmdenv[skipabove=7pt,
skipbelow=4pt,
rightline=true,
leftline=true,
topline=true,
bottomline=true,
linecolor=purple,
backgroundcolor=purple!0,
innerleftmargin=5pt,
innerrightmargin=5pt,
innertopmargin=5pt,
leftmargin=0cm,
rightmargin=0cm,
linewidth=1pt,
innerbottommargin=5pt]{pBox}

% hunter green box
\definecolor{huntergreen}{rgb}{0.21, 0.37, 0.23}
\newmdenv[skipabove=7pt,
skipbelow=4pt,
rightline=true,
leftline=true,
topline=true,
bottomline=true,
linecolor=huntergreen,
backgroundcolor=huntergreen!0,
innerleftmargin=5pt,
innerrightmargin=5pt,
innertopmargin=5pt,
leftmargin=0cm,
rightmargin=0cm,
linewidth=1pt,
innerbottommargin=5pt]{hBox}

% lavender (floral) box
\definecolor{lavender(floral)}{rgb}{0.71, 0.49, 0.86}
\newmdenv[skipabove=7pt,
skipbelow=4pt,
rightline=true,
leftline=true,
topline=true,
bottomline=true,
linecolor=lavender(floral),
backgroundcolor=lavender(floral)!0,
innerleftmargin=5pt,
innerrightmargin=5pt,
innertopmargin=5pt,
leftmargin=0cm,
rightmargin=0cm,
linewidth=1pt,
innerbottommargin=5pt]{lBox}

% air force blue box
\definecolor{airforceblue}{rgb}{0.36, 0.54, 0.66}
\newmdenv[skipabove=7pt,
skipbelow=4pt,
rightline=true,
leftline=true,
topline=true,
bottomline=true,
linecolor=airforceblue,
backgroundcolor=airforceblue!0,
innerleftmargin=5pt,
innerrightmargin=5pt,
innertopmargin=5pt,
leftmargin=0cm,
rightmargin=0cm,
linewidth=1pt,
innerbottommargin=5pt]{aBox}

% teal box
\newmdenv[skipabove=7pt,
skipbelow=4pt,
rightline=true,
leftline=true,
topline=true,
bottomline=true,
linecolor=teal,
backgroundcolor=teal!0,
innerleftmargin=5pt,
innerrightmargin=5pt,
innertopmargin=5pt,
leftmargin=0cm,
rightmargin=0cm,
linewidth=1pt,
innerbottommargin=5pt]{tBox}

\definecolor{satinsheengold}{rgb}{0.8, 0.63, 0.21}
\newmdenv[skipabove=7pt,
skipbelow=4pt,
rightline=true,
leftline=true,
topline=true,
bottomline=true,
linecolor=satinsheengold,
backgroundcolor=satinsheengold!0,
innerleftmargin=5pt,
innerrightmargin=5pt,
innertopmargin=5pt,
leftmargin=0cm,
rightmargin=0cm,
linewidth=1pt,
innerbottommargin=5pt]{sBox}

% dim box
\newmdenv[skipabove=7pt,
skipbelow=4pt,
rightline=false,
leftline=true,
topline=false,
bottomline=false,
linecolor=black,
backgroundcolor=grey245!0,
innerleftmargin=5pt,
innerrightmargin=5pt,
innertopmargin=5pt,
leftmargin=0cm,
rightmargin=0cm,
linewidth=2pt,
innerbottommargin=5pt]{blackBox}

\newenvironment{defn}{\begin{bBox}\begin{definitionT}}{\end{definitionT}\end{bBox}}
\newenvironment{thm}{\begin{gBox}\begin{theoremT}}{\end{theoremT}\end{gBox}}
\newenvironment{coro}{\begin{oBox}\begin{corollaryT}}{\end{corollaryT}\end{oBox}}
\newenvironment{lemma}{\begin{lBox}\begin{lemmaT}}{\end{lemmaT}\end{lBox}}
\newenvironment{rem}{\begin{oBox}\begin{remarkT}}{\end{remarkT}\end{oBox}}
\newenvironment{exa}{\begin{sBox}\begin{exampleT}}{\end{exampleT}\end{sBox}}
\newenvironment{es}{\begin{pBox}\begin{exerciseT}}{\end{exerciseT}\end{pBox}}
\newenvironment{prp}{\begin{hBox}\begin{propertyT}}{\end{propertyT}\end{hBox}}
\newenvironment{home}{\begin{aBox}\begin{homeworkT}}{\end{homeworkT}\end{aBox}}
\newenvironment{subtle}{\begin{tBox}\begin{hintT}}{\end{hintT}\end{tBox}}

\renewcommand{\qed}{\tag*{$\blacksquare$}}
\renewenvironment{proof}{\begin{blackBox}\begin{proofT}}{\[\qed\]\end{proofT}\end{blackBox}}

%---------------------------
% CONTENTS
%---------------------------

\setcounter{secnumdepth}{3} % \subsubsection is level 3
\setcounter{tocdepth}{2}

\usepackage{bookmark}% loads hyperref too
    \hypersetup{
        bookmarksnumbered=true,
        bookmarksopen=true,
        bookmarksopenlevel=1,
        hidelinks,% remove border and color
        pdfstartview=Fit, % Fits the page to the window.
        pdfpagemode=UseOutlines, %Determines how the file is opening in Acrobat; the possibilities are UseNone, UseThumbs (show thumbnails), UseOutlines (show bookmarks), FullScreen, UseOC (PDF 1.5), and UseAttachments (PDF 1.6). If no mode if explicitly chosen, but the bookmarks option is set, UseOutlines is used.
    }

\usepackage{glossaries} % certain packages that must be loaded before glossaries, if they are required: hyperref, babel, polyglossia, inputenc and fontenc
\setacronymstyle{long-short}

% hide section from the ToC \tocless\section{hide}
\newcommand{\nocontentsline}[3]{}
\newcommand{\tocless}[2]{\bgroup\let\addcontentsline=\nocontentsline#1{#2}\egroup}

\usepackage[textsize=tiny, textwidth=1.5cm]{todonotes} % add disable to options to not show in pdf


% =====================================================================================
% Packages Required
% =====================================================================================
%\usepackage{listings}
\usepackage{matlab-prettifier}
\usepackage[most]{tcolorbox}
%\tcbuselibrary{listings,breakable}

% =====================================================================================
% Marker setup
% =====================================================================================

\newtcolorbox{marker}[1][]{enhanced,
    before skip=2mm,after skip=3mm,
    boxrule=0.4pt,left=5mm,right=2mm,top=1mm,bottom=1mm,
    colback=yellow!50,
    colframe=yellow!20!black,
    sharp corners,rounded corners=southeast,arc is angular,arc=3mm,
    underlay={%
        \path[fill=tcbcolback!80!black] ([yshift=3mm]interior.south east)--++(-0.4,-0.1)--++(0.1,-0.2);
        \path[draw=tcbcolframe,shorten <=-0.05mm,shorten >=-0.05mm] ([yshift=3mm]interior.south east)--++(-0.4,-0.1)--++(0.1,-0.2);
        \path[fill=yellow!50!black,draw=none] (interior.south west) rectangle node[white,rotate=0]{\Huge\bfseries ! } ([xshift=4mm]interior.north west);
    },
    drop fuzzy shadow,#1}

%\tcbset{colback=white,colframe=teal,fonttitle=\bfseries,adjusted title=left,breakable=true}
%\begin{tcolorbox}[title=Teorema di unicità]
%Hello
%\end{tcolorbox}

% =====================================================================================
% Listing setup
% =====================================================================================

\lstdefinestyle{myListing}{
    language=Matlab,
    style=Matlab-editor,
    %mlsectiontitlestyle=\scshape\color[RGB]{34,139,34},
    mlsectiontitlestyle=\bfseries\color[RGB]{34,139,34},
    %basicstyle=\small\ttfamily,
    basicstyle=\mlttfamily,
    literate=
    {è}{{\mlttfamily\`e}}1
    {à}{{\mlttfamily\`a}}1
    {ì}{{\mlttfamily\`i}}1
    {ò}{{\mlttfamily\`o}}1
    {ù}{{\mlttfamily\`u}}1
    {-}{{\ttfamily -}}1,
    frame=none,
    keywordstyle=\color[RGB]{0,0,255},
    commentstyle=\color[RGB]{34,139,34},
    stringstyle=\color[RGB]{160,32,240},
    keepspaces,
    %morecomment=[l][\color[RGB]{0,0,255}]{...},
    deletekeywords={%
        size,subplot,plot,xlabel,ylabel,rand,quad,randn,sin,cos,clear,disp,input,fprintf,strcmp,norm,cond,%
        tril,ceil,max,abs,eig,pi,sqrt,round,eye,eps,diag,length,zeros,beta,linspace,end,spdiags,sum,log2,%
    },
}

%\lstMakeShortInline[style=mylisting]"

% =====================================================================================
% matlabcode
% =====================================================================================

\newtcblisting{matlab}{
    top=-2mm,
    bottom=-3mm,
    left=0mm,
    right=0mm,
    boxrule=1.5pt,
    colframe=gray!30!white,
    arc=1mm,
    boxrule=1pt,
    colback=gray!10!white,
    listing only,
    listing options={style=myListing},
    minipage,
    width=\linewidth,
    breakable=true,
}

% =====================================================================================
% matlaboutput
% =====================================================================================

\RequirePackage{verbatim}
\RequirePackage{fancyvrb}
\RequirePackage{color}

\newcommand{\maxwidth}[1]{\ifdim\linewidth>#1 #1\else\linewidth\fi}
\newcommand{\mlcell}[1]{{\color{output}\verbatim@font#1}}

\definecolor{output}{gray}{0.4}

% Unicode character conversions
\DeclareUnicodeCharacter{B0}{\ensuremath{^\circ}}
\DeclareUnicodeCharacter{21B5}{\ensuremath{\hookleftarrow}}

\newenvironment{matlaboutput}{%
        \Verbatim[xleftmargin=1.25em, formatcom=\color{output}]%
}{\endVerbatim}

\usepackage[
	left=3cm, % inner
	right=3cm, % outer
	top=2.5cm,
	bottom=2.5cm,
	%showframe,
	]{geometry}

\usepackage{pdfpages}
\includepdfset{pages=-, offset=0mm 1cm,
   pagecommand={\thispagestyle{empty}}}

\DeclareMathOperator{\Res}{Res}
\DeclareMathOperator{\Log}{Log}
\DeclareMathOperator{\loc}{loc}

%%%%%%%% PARTE ESERCIZI

\usepackage{titlesec}
\usepackage{titletoc}
\usepackage{etoolbox}

\newcommand{\Esercizio}[1]{\subsection{Esercizio~#1}}
\newcommand{\Soluzione}{\subsection{Soluzione}}

\newcommand{\Lezione}{\qquad\hspace*{\fill}\textit}
\newcommand{\Laboratorio}{\qquad\hspace*{\fill}\textit}

\newcommand{\ParteEsercizi}{\section{Esercizi}}
\newcommand{\ParteSoluzioni}{\section{Soluzioni}}

\setcounter{secnumdepth}{2} % mi assicuro che gli esercizi siano numerati
\setcounter{tocdepth}{1} % solo 1 per non avere tutti gli esercizi nel contents

\numberwithin{figure}{section}
\numberwithin{equation}{section}

\begin{document}

\frontmatter

\pagestyle{empty}

% COPERTINA

\hypertarget{mytitlepage}{} % set the hypertarget
\bookmark[dest=mytitlepage,level=chapter]{Title Page} % add the bookmark

\vspace*{\fill}

\begin{center}

	\fg{0.5}{images/logo_polimi}

	\vspace*{1.5cm}

	{\large Notes of}\\
	
	\vspace*{1.5cm}
	
	{\Huge \textsc{Real and Functional Analysis}}\\
	
	\vspace*{1.5cm}
	
	{\large for the Master in Mathematical Engineering}\\
	\vspace*{0.3cm}
	{\large held by Prof. G. Verzini}\\
	\vspace*{0.3cm}
	{\large a.a. 2023/2024}\\

	\vspace*{5cm}

	{\large Edited by}\\
	\vspace*{0.3cm}
	{\large Teo Bonfa}\\

	\vspace*{1.5cm}

	\fg{0.2}{images/logo_blackCL}
\end{center}
\vspace*{\fill}
\clearpage

% INDICE

\cleardoublepage
\pagestyle{toc}
\hypertarget{mytoc}{} % set the hypertarget
\bookmark[dest=mytoc,level=chapter]{\contentsname} % add the bookmark
\tableofcontents
\cleardoublepage

% MAIN MATTER

\pagestyle{fancy}
\mainmatter

\part{Introduction}
%!TEX root = ../main.tex

\setcounter{chapter}{-1}
\chapter{Course structure}
\thispagestyle{empty}

This course is splitted in two parts:
\begin{enumerate}{
	\item Real Analysis $\leadsto$ measure and integration theory, in particular:
	\begin{itemize}{
		\item Collections and sequences of sets
		\item Measurable space, measure, outer measure
		\item Generation of an outer measure
		\item Carathéodory's condition, measure induced by an outer measure
		\item Lebesgue's measure on $\RR^n$
		\item Measurable functions
		\item The Lebesgue integral
		\item Abstract integration
		\item Monotone convergence theorem, Fatou's Lemma, Lebesgue's dominated convergence theorem
		\item Comparison between the Lebesgue and Riemann integrals
		\item Different types of convergence
		\item Derivative of a measure and the Radon-Nikodym theorem
		\item Product measures and the Fubini-Tonelli theorem
		\item Functions of bounded variation and absolutely continuous functions
	}
	\end{itemize}
	\item Functional Analysis $\leadsto$ infinte dimensional linear algebra, in particular:
	\begin{itemize}{
		\item Metric spaces, completeness, separability, compactness
		\item Normed spaces and Banach spaces
		\item Spaces of integrable functions
		\item Linear operators
		\item Uniform boundedness theorem, open mapping theorem, closed graph theorem
		\item Dual spaces and the Hahn-Banach theorem
		\item Reflexivity
		\item Weak and weak* convergences
		\item Banach-Alaoglu theorem
		\item Compact operators
		\item Hilbert spaces
		\item Projection theorem, Riesz representation theorem
		\item Orthonormal basis, abstract Fourier series
		\item Spectral theorem for compact symmetric operators
		\item Fredholm alternativ
	}
	\end{itemize}
}
\end{enumerate}

The foundation of this theory is the \emph{Set Theory}, that is going to be explained in the next chapter. Enjoy!

\bigskip
\bigskip

\textbf{NB:} this page will be updated with more details and maybe the list of proofs. \cleardoublepage
%!TEX root = ../main.tex

\chapter{Set Theory}
\thispagestyle{empty}

\section{Equipotent, finite/infinite, countable/uncountable sets, cardinality of continoum} % (fold)
\label{sec:equipotent_finite_infinite_countable_uncountable_sets_cardinality_of_continoum}

Let $X,Y$ be sets.

\begin{defn}[Equipotent sets]$\\$
$X,Y$ are equipotent if there exists a bijection $f:X\to Y$ (1-1 injective + onto surjective).
\end{defn}

If $X,Y$ are equipotent, then they have the same cardinality. On the other hand, $X$ has cardinality $\geq$ than $Y$ if there exists $f:X\to Y$ onto. For example, for
\begin{equation*}
X= 
\begin{pmatrix}
 1\\
 2\\
 3
\end{pmatrix} \qquad Y=\begin{pmatrix}
 1\\
 2
\end{pmatrix}
\end{equation*}

exists $f:X\to Y$ s.t. $\forall y\in Y\ \exists x\in X$ s.t. $f(x)=y$ ($f$ \emph{takes} all the elements of the codomain), but doesn't exist $g:Y\to X$ s.t. $\forall x\in X\ \exists y\in Y$ s.t. $g(y)=x$ ($g$ doesn't \emph{take} all the elements of the codomain).

\begin{defn}[Finite/infinite sets]$\\$
$X$ is finite if it is equipotent to $Y=\{1,2,...,k\}$ for some $k\in\NN$. $X$ is infinite otherwise.
\end{defn}

\begin{prp}
$X$ is infinite iff it is equipotent to a proper subset, i.e. if exists a bijection between $X$ and one of his subsets.
\end{prp}

For example, between the integers set $\ZZ=\{0,\pm1,\pm2,...\}$ and the even integers set $\{0,\pm2,\pm4,...\}$ there exists $f$ s.t. $f(z)=2z$ which is a bijection.

\begin{defn}[Countable/uncountable (infinite) sets]$\\$
$X$ inifinite is countable if it is equipotent to $\NN$. It is uncountable otherwise, in which case is more than countable (countable sets are the "smallest" among infinite sets).
\end{defn}

\begin{defn}[Cardinality of continoum]$\\$
$X$ has the cardinality of continoum if it is equipotent to $\RR$. Any such set is uncountable.
\end{defn}

For example:
\begin{itemize}
	\item $\NN,\ZZ,\QQ$ are countable
	\item $\RR,\RR^N,(0,1),(0,1)^N$ have the cardinality of continoum
	\item countable unions of countable sets are countable
\end{itemize}

% section equipotent_finite_infinite_countable_uncountable_sets_cardinality_of_continoum (end)

\section{Families of subsets} % (fold)
\label{sec:families_of_subsets}

Let $X$ be a set.

\begin{defn}[Power set]$\\$
The power set of $X$, i.e. the set of all subsets of $X$, is
\begin{equation*}
    \Pc(X)=\gr{Y\,:\,Y\subset X}
\end{equation*}
It is sometimes denoted as $2^X$.
\end{defn}

The power set has cardinality strictly bigger than $X$. For example, $\Pc(\NN)$ has the cardinality of continoum.

\begin{defn}[Family of subsets]$\\$
A family, or collection, of subsets of $X$ is just $\Cc\subset\Pc(X)$. Tipically, a family of subsets (induced by $I\subset\RR$ set of indexes) is $\Cc=\gr{E_i}_{i\in I}$ where $E_i\subset X$ $\forall i\in I$.
\end{defn}

For example, $\left\{E_1,E_2,E_3\right\}$ is a family of subsets.

\begin{defn}[Union and intersection]$\\$
Given a family of sets $\left\lbrace E_i \right\rbrace_{i \in I} \subset \mathcal{P}(X)$, will often be considered
\begin{align*}
\bigcup_{i \in I} E_i &= \gr{x \in X\,:\, \exists i \in I \text{ s.t. } x \in E_i} \\ 
\bigcap_{i \in I} E_i &= \gr{x \in X\,:\,x \in E_i\ \forall i \in I}  
\end{align*}

$\gr{E_i}$ is said to be (pairwise) disjoint if $E_i \cap E_j = \emptyset$ $\forall i \not = j$.
\end{defn}

\begin{exa}[Standard topology of \texorpdfstring{$\RR$}{C}]$\\$
Given $X=\RR$ $\td{\text{or }\RR^N}$, the standard/euclidian topology of $\RR$ $\td{\text{or }\RR^N}$ is $\Tc=\gr{E\subset X\,:\,E\text{ is open}}$, i.e. it is the family of all open subsets of $X$. \\
More generally, this can be defined in metric spaces $\td{X,d}$ where $X$ is a set and $d$ a distance between $x,y\in X$.

Some properties of $\Tc$:
\begin{itemize}
    \item $\varnothing,X\in\Tc$
    \item finite intersection of open sets is open [$\circledast$]
    \item any (finite/infinite, countable/uncountable, ...) union of open sets is open [$\circledcirc$]
\end{itemize}
\end{exa}

\begin{defn}[Covering and subcovering]$\\$
$\gr{E_i}_{i\in I}$ is a covering of $X$ if $X=\bigcup_{i\in I} E_i$. Any subfamily $\gr{E_i}_{i\in J, J\subset I}$ is a subcovering if it is a covering.
\end{defn}

% section families_of_subsets (end)

\newpage

\section{Sequences of sets} % (fold)
\label{sec:sequences_of_sets}

A sequence is just a family of subsets where $I\equiv \NN$, e.g. $\gr{E_n}_{n\in \NN}$. 

\begin{defn}[Monotone sequences]$\\$
$\gr{E_n}$ is increasing (not decreasing), $\gr{E_n}\nearrow$, if $E_n\subset E_{n+1}\ \forall n\in\NN$. On the other hand, $\gr{E_n}$ is decreasing (not increasing), $\gr{E_n}\searrow$, if $E_{n+1}\subset E_{n}\ \forall n\in\NN$. If $\gr{E_n}$ is increasing/decreasing then it is monotone.
\end{defn}

For example, given $X=\RR$ and $E_n\displaystyle=\td{-\frac{1}{n},1+\frac{1}{n}}$ for $n\geq 1$, we can say that $E_n$ is a monotone decreasing sequence:
\fg{0.3}{endrc}

But what is $\bigcap_{n=1}^\infty E_n$? We know that 
\begin{equation*}
    \bigcap_{n=1}^\infty E_n=[0,1]
\end{equation*}

and this is an infinite intersection of open sets (this does not disagree with the prop $\circledast$). This type of intersection is called "G$\delta$-set": a countable intersection of open sets.

Similarly, $E_n=\displaystyle\qu{a+\frac{1}{n},b-\frac{1}{n}}$, a<b, is increasing and 
\begin{equation*}
    \bigcup_{n=1}^\infty E_n=(a,b)
\end{equation*}
is called "F$\sigma$-set": a countable union of closed sets (doesn't disagree with $\circledcirc$).

\begin{defn}[lim sup and lim inf]$\\$
Let $\gr{E_n}_{n\in\NN}\subset\Pc$. We define
\begin{equation*}
    \limsup_{n} E_n := \bigcap_{n = 1}^{\infty} \left(\bigcup_{k = n}^{\infty} E_k\right)
    \qquad
    \liminf_{n} E_n := \bigcup_{n = 1}^{\infty} \left(\bigcap_{k = n}^{\infty} E_k\right)
\end{equation*}

If these two sets are equal
\begin{equation*}
    \limsup_n E_n = \liminf_n E_n = \lim_n E_n = F
\end{equation*}
then $F$ is the limit of the succession.
\end{defn}

Take note that $\gr{E_n}\nearrow \td{\searrow} \Longrightarrow \exists \lim_n E_n=\bigcup_n E_n \td{\bigcap_n E_n}$.

% section sequences_of_sets (end)



































 \cleardoublepage

\part{Real Analysis}
%!TEX root = ../main.tex

\chapter{}
\thispagestyle{empty} \cleardoublepage
%!TEX root = ../main.tex

\chapter{}
\thispagestyle{empty} \cleardoublepage
%!TEX root = ../main.tex

\chapter{}
\thispagestyle{empty} \cleardoublepage
%!TEX root = ../main.tex

\chapter{}
\thispagestyle{empty} \cleardoublepage
%!TEX root = ../main.tex

\chapter{}
\thispagestyle{empty} \cleardoublepage
%!TEX root = ../main.tex

\chapter{}
\thispagestyle{empty} \cleardoublepage
%!TEX root = ../main.tex

\chapter{}
\thispagestyle{empty} \cleardoublepage
%!TEX root = ../main.tex

\chapter{}
\thispagestyle{empty} \cleardoublepage
%!TEX root = ../main.tex

\chapter{Antonino}
\thispagestyle{empty} \cleardoublepage

\part{Functional Analysis}
%!TEX root = ../main.tex

\chapter{}
\thispagestyle{empty} \cleardoublepage
%!TEX root = ../main.tex

\chapter{}
\thispagestyle{empty} \cleardoublepage
%!TEX root = ../main.tex

\chapter{}
\thispagestyle{empty} \cleardoublepage
%!TEX root = ../main.tex

\chapter{}
\thispagestyle{empty} \cleardoublepage
%!TEX root = ../main.tex

\chapter{}
\thispagestyle{empty} \cleardoublepage
%!TEX root = ../main.tex

\chapter{}
\thispagestyle{empty} \cleardoublepage
%!TEX root = ../main.tex

\chapter{}
\thispagestyle{empty} \cleardoublepage
%!TEX root = ../main.tex

\chapter{}
\thispagestyle{empty} \cleardoublepage

\part{Esercitazioni}
% \setcounter{chapter}{0}
\numberwithin{equation}{chapter}
%!TEX root = ../main.tex

\chapter{}
\thispagestyle{empty} \cleardoublepage
%!TEX root = ../main.tex

\chapter{}
\thispagestyle{empty} \cleardoublepage
%!TEX root = ../main.tex

\chapter{}
\thispagestyle{empty} \cleardoublepage
%!TEX root = ../main.tex

\chapter{}
\thispagestyle{empty} \cleardoublepage
%!TEX root = ../main.tex

\chapter{}
\thispagestyle{empty} \cleardoublepage
%!TEX root = ../main.tex

\chapter{}
\thispagestyle{empty} \cleardoublepage
%!TEX root = ../main.tex

\chapter{}
\thispagestyle{empty} \cleardoublepage
%!TEX root = ../main.tex

\chapter{}
\thispagestyle{empty} \cleardoublepage
%!TEX root = ../main.tex

\chapter{}
\thispagestyle{empty} \cleardoublepage
%!TEX root = ../main.tex

\chapter{}
\thispagestyle{empty} \cleardoublepage

\end{document}